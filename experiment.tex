%% Theorie.tex
%%
%\usepackage[ngerman]{babel}
%% ==============
\chapter{Experimentelle Grundlagen}
\label{ch:experiment}
%% ==============

{\bibliographystyle{babalpha-fl}}	% german style

Zur Untersuchung der theoretischen Vorhersagen, werden weltweit Experimente durchgef\"uhrt. In Kapitel \cite{ch:Experiment} wird das Compact Muon Solenoid (CMS) Experiment am Large Hadron Collider (LHC) des CERN vorgestellt. Anschlie\ss end wird der Ablauf einer Hochenergiephysik-Analyse am Beispiel der \ttH-Analyse vorgestellt.

%% ===========================
\section{Der Large Hadron Collider (LHC)}
\label{ch:Experiment:sec:LHC}
%% ===========================

%% ===========================
\section{Das Compact Muon Sollenoid (CMS) Experiment}
\label{ch:Experiment:sec:CMS}
%% ===========================

%% ===========================
\section{\ttH-Analyse}
\label{ch:Experiment:sec:ttH}
%% ===========================
