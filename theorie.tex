%% Theorie.tex
%%
%\usepackage[ngerman]{babel}
%% ==============
\chapter{Theorie}
\label{ch:Theorie}
%% ==============



%% ===========================
\section{Das Standardmodell}
\label{ch:Theorie:sec:Standardmodell}
%% ===========================

Der folgenden Abschnitt soll einen kurzen \"uberblick \"uber das Standardmodell der Elementarteilchenphysik geben, dabei bezieht er sich meist auf \cite{SWB-39819646X}.

Das Standardmodell der Elementarteilchenphysik fasst verschiedene Theorien der Teilchenphysik zusammen und vereinheitlicht diese. Wie im Artikel von Cian O'Luanaigh \cite{O'Luanaigh:1997201} beschrieben, besteht das Universum aus einigen grundlegenden Bausteinen, die durch vier elementare Kr\"afte beeinflusst werden. Die bislang beste Beschreibung dieses Aufbaus liefert das Standardmodell. Mithilfe dieses Modells war es m\"oglich fast alle experimentellen Ergebnisse zu best\"atigen, sowie sehr pr\"azise Vorhersagen \"uber verschiedene Ph\"anomene zu treffen.

Die bislang entdeckte Materie besteht aus zwei Arten von Elementarteilchen, den Leptonen sowie den Quarks. Diese lassen sich jeweils in drei Familien unterteilen. Jede Quark-Familie besteht jeweils aus einem Quarkpaar, diese sind Up- und Down-, Strange- und Charm-, Bottom- und Topquark. Sie unterliegen der starken Wechselwirklung.\\
Leptonen bilden jeweils zusammen mit dem dazugeh\"origen Neutrino eine Familie. Im Gegensatz zu den Quarks unterliegen Leptonen der schwachen Wechselwirkung.\\
Die dritte elementare Wechselwirkung, die im Standardmodell beschrieben ist (die Gravitation kann nicht durch das Standardmodell erkl\"art werden), ist die elektromagnetische Wechselwirkung. Diese Wechselwirkungen sind in ihrer Struktur sehr \"ahnlich und werden durch den Austausch von Vektorbosonen vermittelt. Diese sind die Gluonen der starken Wechselwirkung, die W- und Z-Bosonen der schwachen Wechselwirkung und die Photonen der elektromagnetischen. 

Der letzte fehlende Baustein im Standardmodell ist ein elementares Spin-0 Teilchen, ohne das keine konsistente Erkl\"arung f\"ur die W und Z$^0$ Massen m\"oglich w\"are. Dieses ist das Higgs-Boson, welches 2012 am CERN entdeckt wurde. Die Kopplung zwischen Higgs-Boson und anderen Elementarteilchen ist proportional zur Teilchenmasse.

Insgesamt stimmen die Ergebnisse der Experimente sehr genau mit den Vorhersagen des Standardmodells \"uberein. Dennoch reicht das Modell nicht aus, um s\"amtliche Ph\"anomene zu erkl\"aren. Im Modell werden beispielsweise masselose Neutrinos gefordert, allerdings ist durch die Neutrinooszillationen erwiesen, dass massive Neutrinos existieren.


%% ===========================
\section{Algorithmen zur multivariaten Analyse}
\label{ch:Theorie:sec:Algorithmen}
%% ===========================

Multivariate Analysemethoden bezeichnen Verfahren, mit denen, im Gegensatz zu univariaten Analyse statt jeder Variable einzeln, mehrere Variablen zugleich statistisch untersucht werden.\\
Aufgrund der sehr komplexen Problemstellungen ist eine Berechnung sehr aufw\"andig und daher manuell nicht zu bewerkstelligen. Mithilfe der zunehmenden Rechenleistung aktueller Computer ist dies jedoch m\"oglich und wird in vielen Bereichen immer wichtiger. Die Erforschung und Entwicklung dieser mathematischen Modelle bezeichnet man auch als maschinelles Lernen (machine learning), da man mithilfe der Algorithmen versucht wird, aus Daten zu lernen und Vorhersagen zu treffen. \cite{SWB-455193959}

In der Hochenergiephysik nutzt man diese Methoden, indem man die Algorithmen, mithilfe der, durch theoretische Berechnungen erstellten, Simulationsdaten trainiert und anschlie\ss end die gemessen Daten klassifiziert.\\ 
Maschinelles Lernen spielt abe auch in vielen anderen Bereichen eine wichtige Rolle, wie beispielsweise im Finanzwesen, bei Studien zum Konsumverhalten, oder der Sprach-, Schrift- und Bilderkennung.

F\"ur diese Algorithmen existieren verschiedene Ans\"atze. Einer dieser Ans\"atze ist das \"uberwachte Lernen (supervised learning). Dabei wird anhand bekannter Trainingseingabewerte eine Klassifizierung der unbekannten Messwerte vorgenommen. Beispiele sind die St\"utzvektormethode, wobei jedoch die englische Bezeichnung support vector machine (SVM) gebr\"auchlich ist, Random Forest (RF), was Zuf\"alliger Wald bedeutet und mehrere zuf\"allig erstellte Entscheidungsb\"aume bezeichnet, oder Neuronale Netze. Ein weiteres Beispiel sind verst\"arkte Entscheidungsb\"aume (Boosted Decision Trees (BDTs)), die im Gegensatz zu Random Forests keine unkorrelierten Entscheidungsb\"aume nutzen.



\subsection{Boosted Decision Trees (BDTs)}
\label{ch:Algorithmen:subsec:BDT}
%% ===========================