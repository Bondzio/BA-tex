%% Theorie.tex
%%
%\usepackage[ngerman]{babel}
%% ==============
\chapter{Theorie}
\label{ch:Theorie}
%% ==============



%% ===========================
\section{Das Standardmodell}
\label{ch:Theorie:sec:Standardmodell}
%% ===========================

Der folgenden Abschnitt soll einen kurzen \"uberblick \"uber das Standardmodell der Elementarteilchenphysik geben, dabei bezieht er sich meist auf \cite{SWB-39819646X}.

Das Standardmodell der Elementarteilchenphysik fasst verschiedene Theorien der Teilchenphysik zusammen und vereinheitlicht diese. Wie im Artikel von Cian O'Luanaigh \cite{O'Luanaigh:1997201} beschrieben, besteht das Universum aus einigen grundlegenden Bausteinen, die durch vier elementare Kr\"afte beeinflusst werden. Die bislang beste Beschreibung dieses Aufbaus liefert das Standardmodell. Mithilfe dieses Modells war es m\"uglich fast alle experimentellen Ergebnisse zu best\"atigen, sowie sehr pr\"azise Vorhersagen \"uber verschiedene Ph\"anomene zu treffen.

Die bislang entdeckte Materie besteht aus zwei Arten von Elementarteilchen, den Leptonen sowie den Quarks. Diese lassen sich jeweils in drei Familien unterteilen. Jede Quark-Familie besteht jeweils aus einem Quarkpaar, diese sind Up- und Down-, Strange- und Charm-, Bottom- und Topquark. Sie unterliegen der starken Wechselwirklung.\\
Leptonen bilden jeweils zusammen mit dem dazugeh\"origen Neutrino eine Familie. Im Gegensatz zu den Quarks unterliegen Leptonen der schwachen Wechselwirkung.\\
Die dritte elementare Wechselwirkung, die im Standardmodell beschrieben ist (die Gravitation kann nicht durch das Standardmodell erkl\"art werden), ist die elektromagnetische Wechselwirkung. Diese Wechselwirkungen sind in ihrer Struktur sehr \"ahnlich und werden durch den Austausch von Vektorbosonen vermittelt. Diese sind die Gluonen der starken Wechselwirkung, die W- und Z-Bosonen der schwachen Wechselwirkung und die Photonen der elektromagnetischen. 


%% ===========================
\section{Ertellen von Tabellen}
\label{ch:Kapitel1:sec:Tabellen}
%% ===========================

\begin{verbatim}
\begin{table}[hhh]\parbox{12cm}{
  \caption[Physical constants]
  {\it Beisspiel: Tabelle physikalischer  Konstanten{\rm \cite{Beringer:1900zz}}
  }\label{tab:physconst}}
  \begin{tabular}{lll}
  \hline
  {\bf Symbol} & {\bf Beschreibung} & {\bf Einheit}  \\
  \hline \hline
     $c$    & Lichtgeschwindigkeit  &  $299792458$~ms$^{-1}$ \\
     $h$    & Planck Konstante & $6.626098 \times 10^{-34}$~Js \\
     $\hbar = \frac{h}{2 \pi}$ & Planck Konstante, reduziert &
            $1.054571 \times 10^{-34}$~Js \\
     $e$    & elektrische Elementarladung & $1.602176 \times 10^{-19}~C$ \\ 
  $\epsilon_0$ & elektrische Feldkontante & 
                  $8.854187 \times 10^{-12}$~Fm$^{-1}$ \\
  $\mu_0$       & magnetische Feldkonstante & $4\pi \times 
          10^{-7}$~NA$^{-2} = 12.566270 \times 10^{-7}$~NA$^{-2}$ \\
   $m_e$  & Elektronen Masse             &  $0.510998$~MeV/c$^2$\\
   $m_p$  & Protonen Masse               &  $938.271998$~MeV/c$^2$ \\        
   $N_A$  & Avogadro Zahl         & $6.022144 \cdot 
                                        10^{23}~{\rm mol}^-1 $ \\
   $r_e$  & Elektronenradius & $2.817940 \cdot 10^{-13}$~cm \\                                     
  \hline
  \end{tabular}
\end{table}
\end{verbatim}

\begin{table}[hhh]\parbox{12cm}{
  \caption[Physical constants]{\it Beisspiel: Tabelle physikalischer  Konstanten{\rm \cite{Beringer:1900zz}}
  }\label{tab:physconst}}
  \begin{tabular}{lll}
  \hline
  {\bf Symbol} & {\bf Beschreibung} & {\bf Einheit}  \\
  \hline \hline
     $c$    & Lichtgeschwindigkeit  &  $299792458$~ms$^{-1}$ \\
     $h$    & Planck Konstante & $6.626098 \times 10^{-34}$~Js \\
     $\hbar = \frac{h}{2 \pi}$ & Planck Konstante, reduziert &
            $1.054571 \times 10^{-34}$~Js \\
     $e$    & elektrische Elementarladung & $1.602176 \times 10^{-19}~C$ \\ 
  $\epsilon_0$ & elektrische Feldkontante & 
                  $8.854187 \times 10^{-12}$~Fm$^{-1}$ \\
  $\mu_0$       & magnetische Feldkonstante & $4\pi \times 
          10^{-7}$~NA$^{-2} = 12.566270 \times 10^{-7}$~NA$^{-2}$ \\
   $m_e$  & Elektronen Masse             &  $0.510998$~MeV/c$^2$\\
   $m_p$  & Protonen Masse               &  $938.271998$~MeV/c$^2$ \\        
   $N_A$  & Avogadro Zahl         & $6.022144 \cdot 
                                        10^{23}~{\rm mol}^-1 $ \\
   $r_e$  & Elektronenradius & $2.817940 \cdot 10^{-13}$~cm \\                                     
  \hline
  \end{tabular}
\end{table}



\dots


%% ===========================
\section{Benutzung n\"utzlicher Pakete}
\label{ch:Kapitel1:sec:Pakete}
%% ===========================

%% ===========================
\subsection{SI Einheiten: siunitx}
\label{ch:Kapitel1:subsec:SiUnit}
%% ===========================

Das Paket siunitx bietet die M\"oglichkeit Einheiten und Zahlen
schnell und einfach darzustellen.\\
Beispiel Code:
\begin{verbatim}
\si{kg.m.s^{-1}} \\
\si{\kilogram\metre\per\second} \\
\si[per-mode=symbol]{\kilogram\metre\per\second} \\
\si[per-mode=symbol]{\kilogram\metre\per\ampere\per\second}
\end{verbatim}
Ausgabe im Text''\\
\si{kg.m.s^{-1}} \\
\si{\kilogram\metre\per\second} \\
\si[per-mode=symbol]{\kilogram\metre\per\second} \\
\si[per-mode=symbol]{\kilogram\metre\per\ampere\per\second}


Eine Einf\"uhrung in das Paket findet man unter:
\begin{verbatim}
ftp://ftp.dante.de/tex-archive/macros/latex/exptl/siunitx/siunitx.pdf
\end{verbatim}

%% ===========================
\subsection{amsmath}
\label{ch:Kapitel1:subsec:amsmath}
%% ===========================

\begin{verbatim}
ftp://ftp.ams.org/pub/tex/doc/amsmath/short-math-guide.pdf
\end{verbatim}

%% ===========================
\subsection{xxx}
\label{ch:Kapitel1:subsec:xxx}
%% ===========================