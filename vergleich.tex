%% Theorie.tex
%%
%\usepackage[ngerman]{babel}
%% ==============
\chapter{Vergleich der multivariaten Algorithmen}
\label{ch:vergleich}
%% ==============

{\bibliographystyle{babalpha-fl}}	% german style

In diesem Kapitel wird zun\"achst erl\"autert, anhand welcher Kriterien die verwendeten Algorithmen miteinander verglichen werden k\"onnen. Anschlie\ss end werden verschiedene Datens\"atze mithilfe der Algorithmen untersucht und die Ergebnisse verglichen.

%% ===========================
\section{Vergleichbarkeit der Algorithmen}
\label{ch:Vergleich:sec:Vergleichbarkeit}
%% ===========================

Bevor die verschiedenen Implementationen der Algorithmen verglichen werden k\"onnen, m\"ussen zun\"achst einige Vergleichskriterien festgelegt werden und \"uberpr\"uft werden inwieweit sich die Parameter der Algorithmen unterscheiden.\\
In Tabelle \ref{tab:parameter} sind die Einstellungsm\"oglichkeiten der drei Algorithmen dargestellt.

\begin{table}[hhh]\parbox{12cm}{
  \caption[Algorithmenparameter]{\it Tabelle mit einstellbaren Parametern der verschiedenen Algorithmen}% {\rm \cite{Agashe:2014kda}}
  }\label{tab:parameter}
  \begin{center}
  \begin{tabular}{llll}
  \hline
  {\bf TMVA} & {\bf scikit-learn} & {\bf XGBoost} & {\bf Funktion} \\
  \hline \hline
     NTrees	& n\_estimators & n\_estimators & Anzahl der Entscheidungsb\"aume \\
     Shrinkage	& learning\_rate & learning\_rate & Lernrate des Gradient Boosting \\
     MaxDepth & max\_depth & max\_depth & Tiefe der Entscheidungsb\"aume\\
     nCuts & -- & -- & Anzahl an getesteten Schnitten\\ 
  	 MinNodeSize & min_samples_leaf & -- & Minimalanzahl Ereignisse pro Knoten\\                             
  \hline
  \end{tabular}
  \end{center}
\end{table}


%% ===========================
\section{Verwendete Datens\"atze}
\label{ch:Vergleich:sec:Daten}
%% ===========================

%% ===========================
\section{Anwendung und Vergleich der Algorithmen zur ttH Analyse}
\label{ch:Vergleich:sec:ttH}
%% ===========================
