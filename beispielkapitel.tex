%% beispielkapitel.tex
%%

%% ==============
\chapter{Kapitel 1}
\label{ch:Kapitel1}
%% ==============

Der Titel der Kapitel sollte entsprechend angepasst werden, z.B. 
Kpitel 1 nach 'Das CMS Experiment' und entsprechend die Abschnitte 
umbenannt werden z.B. 'Der Silizumstreifen Detektor' oder 'Der Pixel 
Detektor'.  

Es ist eine gute Idee ein eigenes File f\"ur jedes Kapitel anzulegen.

%% ===========================
\section{Einbinden von Grafiken}
\label{ch:Kapitel1:sec:Grafiken}
%% ===========================

Grafiken werden mit 'inlcudegrapghics' eigebunden, f\"ur hier
benutzte Vorlage sollten 'pdf' Grafiken eingebunden werden.  

\begin{verbatim}
\begin{figure}[hhh]
 \begin{center}
   \includegraphics[width=.3\textwidth]{logos/KITLogo_RGB.pdf}
   \parbox[b]{12cm}{
     \caption[Beispiel Abbildung (kurze Unterschrift f\"ur Abbildungsverzeichnis)]
             {\label{fig:exmaple} \it Beispiel Abbildung
               (ausf\"uhrliche Bildunterschrift, die unter dem Bild
               angezeigt wird)}
   }
 \end{center}
\end{figure}
\end{verbatim}

\begin{figure}[hhh]
 \begin{center}
   \includegraphics[width=.3\textwidth]{logos/KITLogo_RGB.pdf}
   \parbox[b]{12cm}{
     \caption[Beispiel Abbildung (kurze Unterschrift f\"ur Abbildungsverzeichnis)]
             {\label{fig:exmaple} \it Beispiel Abbildung
               (ausf\"uhrliche Bildunterschrift, die unter dem Bild
               angezeigt wird)}
   }
 \end{center}
\end{figure}




\dots


%% ===========================
\section{Ertellen von Tabellen}
\label{ch:Kapitel1:sec:Tabellen}
%% ===========================

\begin{verbatim}
\begin{table}[hhh]\parbox{12cm}{
  \caption[Physical constants]
  {\it Beisspiel: Tabelle physikalischer  Konstanten{\rm \cite{Beringer:1900zz}}
  }\label{tab:physconst}}
  \begin{tabular}{lll}
  \hline
  {\bf Symbol} & {\bf Beschreibung} & {\bf Einheit}  \\
  \hline \hline
     $c$    & Lichtgeschwindigkeit  &  $299792458$~ms$^{-1}$ \\
     $h$    & Planck Konstante & $6.626098 \times 10^{-34}$~Js \\
     $\hbar = \frac{h}{2 \pi}$ & Planck Konstante, reduziert &
            $1.054571 \times 10^{-34}$~Js \\
     $e$    & elektrische Elementarladung & $1.602176 \times 10^{-19}~C$ \\ 
  $\epsilon_0$ & elektrische Feldkontante & 
                  $8.854187 \times 10^{-12}$~Fm$^{-1}$ \\
  $\mu_0$       & magnetische Feldkonstante & $4\pi \times 
          10^{-7}$~NA$^{-2} = 12.566270 \times 10^{-7}$~NA$^{-2}$ \\
   $m_e$  & Elektronen Masse             &  $0.510998$~MeV/c$^2$\\
   $m_p$  & Protonen Masse               &  $938.271998$~MeV/c$^2$ \\        
   $N_A$  & Avogadro Zahl         & $6.022144 \cdot 
                                        10^{23}~{\rm mol}^-1 $ \\
   $r_e$  & Elektronenradius & $2.817940 \cdot 10^{-13}$~cm \\                                     
  \hline
  \end{tabular}
\end{table}
\end{verbatim}

\begin{table}[hhh]\parbox{12cm}{
  \caption[Physical constants]{\it Beisspiel: Tabelle physikalischer  Konstanten{\rm \cite{Beringer:1900zz}}
  }\label{tab:physconst}}
  \begin{tabular}{lll}
  \hline
  {\bf Symbol} & {\bf Beschreibung} & {\bf Einheit}  \\
  \hline \hline
     $c$    & Lichtgeschwindigkeit  &  $299792458$~ms$^{-1}$ \\
     $h$    & Planck Konstante & $6.626098 \times 10^{-34}$~Js \\
     $\hbar = \frac{h}{2 \pi}$ & Planck Konstante, reduziert &
            $1.054571 \times 10^{-34}$~Js \\
     $e$    & elektrische Elementarladung & $1.602176 \times 10^{-19}~C$ \\ 
  $\epsilon_0$ & elektrische Feldkontante & 
                  $8.854187 \times 10^{-12}$~Fm$^{-1}$ \\
  $\mu_0$       & magnetische Feldkonstante & $4\pi \times 
          10^{-7}$~NA$^{-2} = 12.566270 \times 10^{-7}$~NA$^{-2}$ \\
   $m_e$  & Elektronen Masse             &  $0.510998$~MeV/c$^2$\\
   $m_p$  & Protonen Masse               &  $938.271998$~MeV/c$^2$ \\        
   $N_A$  & Avogadro Zahl         & $6.022144 \cdot 
                                        10^{23}~{\rm mol}^-1 $ \\
   $r_e$  & Elektronenradius & $2.817940 \cdot 10^{-13}$~cm \\                                     
  \hline
  \end{tabular}
\end{table}



\dots


%% ===========================
\section{Benutzung n\"utzlicher Pakete}
\label{ch:Kapitel1:sec:Pakete}
%% ===========================

%% ===========================
\subsection{SI Einheiten: siunitx}
\label{ch:Kapitel1:subsec:SiUnit}
%% ===========================

Das Paket siunitx bietet die M\"oglichkeit Einheiten und Zahlen
schnell und einfach darzustellen.\\
Beispiel Code:
\begin{verbatim}
\si{kg.m.s^{-1}} \\
\si{\kilogram\metre\per\second} \\
\si[per-mode=symbol]{\kilogram\metre\per\second} \\
\si[per-mode=symbol]{\kilogram\metre\per\ampere\per\second}
\end{verbatim}
Ausgabe im Text''\\
\si{kg.m.s^{-1}} \\
\si{\kilogram\metre\per\second} \\
\si[per-mode=symbol]{\kilogram\metre\per\second} \\
\si[per-mode=symbol]{\kilogram\metre\per\ampere\per\second}


Eine Einf\"uhrung in das Paket findet man unter:
\begin{verbatim}
ftp://ftp.dante.de/tex-archive/macros/latex/exptl/siunitx/siunitx.pdf
\end{verbatim}

%% ===========================
\subsection{amsmath}
\label{ch:Kapitel1:subsec:amsmath}
%% ===========================

\begin{verbatim}
ftp://ftp.ams.org/pub/tex/doc/amsmath/short-math-guide.pdf
\end{verbatim}

%% ===========================
\subsection{xxx}
\label{ch:Kapitel1:subsec:xxx}
%% ===========================